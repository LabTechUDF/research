%
% Documento: Resumo (Inglês)
%

\begin{ABSTRACT}
\thispagestyle{empty}
\OnehalfSpacing

\noindent This Final Undergraduate Thesis presents the development of a Retrieval-Augmented Generation (RAG) pipeline designed to support queries regarding information related to the Unified Electronic Penal Execution System (SEEU) of the National Council of Justice. The objective is to propose an architecture that facilitates the retrieval of support information for different user profiles by integrating vector databases with a Large Language Model (LLM) to provide natural-language responses grounded in original documents. The methodology includes: (i) automated collection of official documents, legislation, and case law; (ii) preprocessing with cleaning and text segmentation; (iii) embedding-based vectorization and indexing in a semantic search engine; (iv) orchestration via the LangChain library between the vector retrieval engine and the LLM; and (v) deployment of the service via chatbot and REST API in a development environment. Expected outcomes include improved information retrieval efficiency, enhanced response quality, and alignment with the Justice 4.0 initiative and UN SDG 16, strengthening transparency and access to justice. It is concluded that the proposed architecture offers a conceptual foundation for the digital transformation of penal execution and provides a basis for future expansions within the Judiciary.

\SingleSpacing
\noindent \textbf{Keywords}: penal execution; vector databases; digital transformation.

\end{ABSTRACT}
