%
% Documento: Resumo (Português)
%

\begin{RESUMO}
\thispagestyle{empty}
\OnehalfSpacing

\noindent Este Trabalho de Conclusão de Curso apresenta o desenvolvimento de uma \textit{pipeline} de Geração Aumentada por Recuperação (\textit{Retrieval-Augmented Generation} -- RAG) concebida para apoiar consultas sobre informações relacionadas ao Sistema Eletrônico de Execução Unificado (SEEU) do Conselho Nacional de Justiça. O objetivo é propor uma arquitetura que facilite a obtenção de informações de suporte pelos diferentes perfis de usuário, integrando bancos de dados vetoriais a um modelo de linguagem de grande porte (\textit{Large Language Model} -- LLM), de modo a fornecer respostas em linguagem natural fundamentadas em documentos originais. A metodologia abrange: (i) coleta automatizada de documentos oficiais, legislação e jurisprudência; (ii) pré-processamento com limpeza e segmentação textual; (iii) vetorização por \textit{embeddings} e indexação em mecanismo de busca semântica; (iv) orquestração, por meio da biblioteca LangChain, entre o motor de recuperação vetorial e o LLM; e (v) disponibilização do serviço por meio de \textit{chatbot} e API REST em ambiente de desenvolvimento. Os resultados esperados incluem ganho de eficiência na recuperação de informações, melhoria da qualidade das respostas e alinhamento às iniciativas Justiça 4.0 e Objetivo de Desenvolvimento Sustentável 16, fortalecendo a transparência e o acesso à Justiça. Conclui-se que a arquitetura proposta oferece base conceitual para a transformação digital da execução penal e constitui fundamento para expansões futuras no âmbito do Poder Judiciário.

\SingleSpacing
\noindent \textbf{Palavras-chaves}: execução penal; bancos de dados vetoriais; transformação digital.

\end{RESUMO}

